%%%%%%%%%%%%%%%%%%%%%%%%%%%%%%%%%%%%%%%%%%%%%%%%%%%%%%%%%%%%%%%%%%%%%%%%%%%%
%                                                                          %
%              (C) Copyright 1995 The Board of Trustees of the             %
%                          University of Illinois                          %
%                           All Rights Reserved                            %
%								  	   %
%%%%%%%%%%%%%%%%%%%%%%%%%%%%%%%%%%%%%%%%%%%%%%%%%%%%%%%%%%%%%%%%%%%%%%%%%%%%

\section{\NAMD\ Availability and Installation}
\label{section:avail}

\NAMD\ is distributed freely for non-profit use.
\NAMD\ \NAMDVER\ is based on the Charm++ messaging system and the
Converse communication layer (\url{http://charm.cs.uiuc.edu/})
which have been ported to a wide variety of parallel platforms.
This section describes how to obtain and install \NAMD\ \NAMDVER.

\subsection{How to obtain \NAMD}

\NAMD\ may be downloaded from \url{http://www.ks.uiuc.edu/Research/namd/}.
You will be required to provide minimal registration information and
agree to a license before receiving access to the software.
Both source and binary distributions are available.

\subsection{Platforms on which \NAMD\ will currently run}
\NAMD\ should be portable to any parallel platform with a
modern C++ compiler to which Charm and Converse have been ported.
Precompiled \NAMD\ \NAMDVER\ binaries are available for
download for the following platforms:  

\begin{itemize}
\item Windows (7, 8, 10, etc.) on x86-64 processors
\item Mac OS X on Intel processors
\item Linux on x86-64 processors
\item Windows, Mac OS X, or Linux with NVIDIA GPUs (CUDA)
\end{itemize}

\NAMD\ may be compiled for the following additional platforms:

\begin{itemize}
\item Cray XT/XE/XK/XC
\item IBM Blue Gene L/P/Q
\item Linux or AIX on POWER processors
\item Linux on ARM processors
\item Linux on ARM or POWER processors with NVIDIA GPUs (CUDA)
\item Linux on x86-64 processors with Intel Xeon Phi coprocessors (MIC)
\end{itemize}

\subsection{Installing \NAMD}

A NAMD binary distribution need only be untarred or unzipped and can
be run directly in the resulting directory.  When building from source
code, ``make release'' will generate a
self-contained directory and .tar.gz or .zip archive that can be moved
to the desired installation location.  Windows and CUDA builds include
Tcl .dll and CUDA .so files that must be in the dynamic library path.

\subsection{Compiling \NAMD}

We provide complete and optimized binaries for all
common platforms to which NAMD has been ported.
It should not be necessary for you to compile
NAMD unless you wish to add or modify features
or to improve performance by using an MPI library
that takes advantage of special networking hardware.

Directions for compiling NAMD are contained in the release notes,
which are available from the NAMD web site
\url{http://www.ks.uiuc.edu/Research/namd/}
and are included in all distributions.

\subsection{Documentation}

All available \NAMD\ documentation is available for download without
registration via the \NAMD\ web site
\url{http://www.ks.uiuc.edu/Research/namd/}.

